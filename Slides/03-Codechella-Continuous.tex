\documentclass[aspectratio=43,t]{beamer}
% xcolor and define colors -------------------------
\usepackage[table]{xcolor}

% https://www.viget.com/articles/color-contrast/
\definecolor{purple}{HTML}{5601A4}
\definecolor{navy}{HTML}{0D3D56}
\definecolor{ruby}{HTML}{9a2515}
\definecolor{alice}{HTML}{107895}
\definecolor{daisy}{HTML}{EBC944}
\definecolor{coral}{HTML}{F26D21}
\definecolor{kelly}{HTML}{829356}
\definecolor{cranberry}{HTML}{E64173}
\definecolor{jet}{HTML}{131516}
\definecolor{asher}{HTML}{555F61}
\definecolor{slate}{HTML}{314F4F}

% Mixtape Sessions
\definecolor{picton-blue}{HTML}{00b7ff}
\definecolor{violet-red}{HTML}{ff3881}
\definecolor{sun}{HTML}{ffaf18}
\definecolor{electric-violet}{HTML}{871EFF}

% Main theme colors
\definecolor{accent}{HTML}{00b7ff}
\definecolor{accent2}{HTML}{871EFF}
\definecolor{gray100}{HTML}{f3f4f6}
\definecolor{gray800}{HTML}{1F292D}


% Beamer Options -------------------------------------

% Background
\setbeamercolor{background canvas}{bg = white}

% Change text margins
\setbeamersize{text margin left = 15pt, text margin right = 15pt}

% \alert
\setbeamercolor{alerted text}{fg = accent2}

% Frame title
\setbeamercolor{frametitle}{bg = white, fg = jet}
\setbeamercolor{framesubtitle}{bg = white, fg = accent}
\setbeamerfont{framesubtitle}{size = \small, shape = \itshape}

% Block
\setbeamercolor{block title}{fg = white, bg = accent2}
\setbeamercolor{block body}{fg = gray800, bg = gray100}

% Title page
\setbeamercolor{title}{fg = gray800}
\setbeamercolor{subtitle}{fg = accent}

%% Custom \maketitle and \titlepage
\setbeamertemplate{title page}
{
    %\begin{centering}
        \vspace{20mm}
        {\Large \usebeamerfont{title}\usebeamercolor[fg]{title}\inserttitle}\\
        {\large \itshape \usebeamerfont{subtitle}\usebeamercolor[fg]{subtitle}\insertsubtitle}\\ \vspace{10mm}
        {\insertauthor}\\
        {\color{asher}\small{\insertdate}}\\
    %\end{centering}
}

% Table of Contents
\setbeamercolor{section in toc}{fg = accent!70!jet}
\setbeamercolor{subsection in toc}{fg = jet}

% Button
\setbeamercolor{button}{bg = accent}

% Remove navigation symbols
\setbeamertemplate{navigation symbols}{}

% Table and Figure captions
\setbeamercolor{caption}{fg=jet!70!white}
\setbeamercolor{caption name}{fg=jet}
\setbeamerfont{caption name}{shape = \itshape}

% Bullet points

%% Fix left-margins
\settowidth{\leftmargini}{\usebeamertemplate{itemize item}}
\addtolength{\leftmargini}{\labelsep}

%% enumerate item color
\setbeamercolor{enumerate item}{fg = accent}
\setbeamerfont{enumerate item}{size = \small}
\setbeamertemplate{enumerate item}{\insertenumlabel.}

%% itemize
\setbeamercolor{itemize item}{fg = accent!70!white}
\setbeamerfont{itemize item}{size = \small}
\setbeamertemplate{itemize item}[circle]

%% right arrow for subitems
\setbeamercolor{itemize subitem}{fg = accent!60!white}
\setbeamerfont{itemize subitem}{size = \small}
\setbeamertemplate{itemize subitem}{$\rightarrow$}

\setbeamertemplate{itemize subsubitem}[square]
\setbeamercolor{itemize subsubitem}{fg = jet}
\setbeamerfont{itemize subsubitem}{size = \small}


% Special characters

\usepackage{collectbox}

\makeatletter
\newcommand{\mybox}{%
    \collectbox{%
        \setlength{\fboxsep}{1pt}%
        \fbox{\BOXCONTENT}%
    }%
}
\makeatother





% Links ----------------------------------------------

\usepackage{hyperref}
\hypersetup{
  colorlinks = true,
  linkcolor = accent2,
  filecolor = accent2,
  urlcolor = accent2,
  citecolor = accent2,
}


% Line spacing --------------------------------------
\usepackage{setspace}
\setstretch{1.1}


% \begin{columns} -----------------------------------
\usepackage{multicol}


% Fonts ---------------------------------------------
% Beamer Option to use custom fonts
\usefonttheme{professionalfonts}

% \usepackage[utopia, smallerops, varg]{newtxmath}
% \usepackage{utopia}
\usepackage[sfdefault,light]{roboto}

% Small adjustments to text kerning
\usepackage{microtype}



% Remove annoying over-full box warnings -----------
\vfuzz2pt
\hfuzz2pt


% Table of Contents with Sections
\setbeamerfont{myTOC}{series=\bfseries, size=\Large}
\AtBeginSection[]{
        \frame{
            \frametitle{Roadmap}
            \tableofcontents[current]
        }
    }


% Tables -------------------------------------------
% Tables too big
% \begin{adjustbox}{width = 1.2\textwidth, center}
\usepackage{adjustbox}
\usepackage{array}
\usepackage{threeparttable, booktabs, adjustbox}

% Fix \input with tables
% \input fails when \\ is at end of external .tex file
\makeatletter
\let\input\@@input
\makeatother

% Tables too narrow
% \begin{tabularx}{\linewidth}{cols}
% col-types: X - center, L - left, R -right
% Relative scale: >{\hsize=.8\hsize}X/L/R
\usepackage{tabularx}
\newcolumntype{L}{>{\raggedright\arraybackslash}X}
\newcolumntype{R}{>{\raggedleft\arraybackslash}X}
\newcolumntype{C}{>{\centering\arraybackslash}X}

% Figures

% \imageframe{img_name} -----------------------------
% from https://github.com/mattjetwell/cousteau
\newcommand{\imageframe}[1]{%
    \begin{frame}[plain]
        \begin{tikzpicture}[remember picture, overlay]
            \node[at = (current page.center), xshift = 0cm] (cover) {%
                \includegraphics[keepaspectratio, width=\paperwidth, height=\paperheight]{#1}
            };
        \end{tikzpicture}
    \end{frame}%
}

% subfigures
\usepackage{subfigure}


% Highlight slide -----------------------------------
% \begin{transitionframe} Text \end{transitionframe}
% from paulgp's beamer tips
\newenvironment{transitionframe}{
    \setbeamercolor{background canvas}{bg=accent!40!black}
    \begin{frame}\color{accent!10!white}\LARGE\centering
}{
    \end{frame}
}


% Table Highlighting --------------------------------
% Create top-left and bottom-right markets in tabular cells with a unique matching id and these commands will outline those cells
\usepackage[beamer,customcolors]{hf-tikz}
\usetikzlibrary{calc}
\usetikzlibrary{fit,shapes.misc}

% To set the hypothesis highlighting boxes red.
\newcommand\marktopleft[1]{%
    \tikz[overlay,remember picture]
        \node (marker-#1-a) at (0,1.5ex) {};%
}
\newcommand\markbottomright[1]{%
    \tikz[overlay,remember picture]
        \node (marker-#1-b) at (0,0) {};%
    \tikz[accent!80!jet, ultra thick, overlay, remember picture, inner sep=4pt]
        \node[draw, rectangle, fit=(marker-#1-a.center) (marker-#1-b.center)] {};%
}

\usepackage{math}
\begin{document}
\imageframe{lecture_includes/slide_banner.pdf}
\section{Continuous Treatment}

\begin{frame}{Setup}
  Individuals are observed in two periods ($t = 0, 1$)

  \bigskip
  Treatment turns on in the second period for some units
  \begin{itemize}
    \item The untreated group receives no dose, $D_i = 0$

    \item The treated group receives a dose $D_i > 0$
  \end{itemize}

  \bigskip
  How do you do difference-in-differences in this setup?
\end{frame}

% \begin{frame}{Aside: Alternate `continuous treatment'}
%   Alternatively, you might have a setting where you have some key variable $D_{it}$ that can vary over time.
%   \begin{itemize}
%     \item E.g. minimum wage changes
%   \end{itemize}
%
%   \bigskip
%   Happy to discuss, but going to focus on the former
%
%   % TODO: Discuss this more; can give tariff example of "transforming into the above setup"
% \end{frame}

\subsection{Discrete multi-valued}

\begin{frame}{Initial Example}
  Let's work through an example to illustrate what's novel in this setting...

  \bigskip
  There are a set of patients who sign up for an experimental medicine.

  They are either not accepted (receive 0mg) or accepted into the program and allowed a positive dose
  \begin{itemize}
    \item Doctor drugs 0mg, 10mg, 20mg
  \end{itemize}

  % TODO: Discuss parallel trends
\end{frame}

\begin{frame}{Estimating effects for each dose group}
  We have three groups:
  \begin{enumerate}
    \item The `untreated' group who received 0mg

    \item The `10mg' group

    \item The `20mg' group
  \end{enumerate}

  \bigskip
  The simplest thing to do is to treat the `10mg' and `20mg' group as two treatments and compare both to the `untreated' group
\end{frame}

\begin{frame}{First-differences}
  The rest of the slides will focus on \emph{first-differences}:
  $$
    \Delta y_i \equiv y_{i1} - y_{i0}
  $$

  \bigskip
  Parallel counterfactual trends assumption says that $\Delta y_i$ would be on average the same for the `10mg' and the untreated group \emph{in the absence of treatment}
  \begin{itemize}
    \item Similarly when looking at the `20mg' group
  \end{itemize}
\end{frame}

\imageframe{lecture_includes/continuous_did/discrete_raw_first_differences.pdf}

\begin{frame}{Diff-in-Diff}
  Our diff-in-diff estimate is formed as:
  $$
    \hat{\tau}_{10} = \expec{\Delta Y_i}{D_i = 10} - \expec{\Delta Y_i}{D_i = 0}
  $$
  and
  $$
    \hat{\tau}_{20} = \expec{\Delta Y_i}{D_i = 20} - \expec{\Delta Y_i}{D_i = 0}
  $$

  \bigskip
  In both cases, we use the untreated group to estimate the counterfactual trend
  \begin{itemize}
    \item $\expec{\Delta Y_i}{D_i = 0}$ is what we think the change in outcomes would have been in the absence of treatment for the treated unit
  \end{itemize}
\end{frame}

\imageframe{lecture_includes/continuous_did/discrete_implied_counterfactual.pdf}
\imageframe{lecture_includes/continuous_did/discrete_did_estimates.pdf}

\begin{frame}{Estimates}
  The previous figure shows two difference estimates of $\hat{\tau}_{10}$ and $\hat{\tau}_{20}$
  \begin{itemize}
    \item The 20mg group has a higher estimated treatment effect than the 10mg group. What can we conclude from that?
  \end{itemize}
\end{frame}

\begin{frame}{ATT Estimates}
  To understand this, let's remember the discussion of ATT vs. ATE

  \begin{itemize}
    \item In the binary DID case, we estimate the average treatment effect \emph{among those who received treatment}
  \end{itemize}

  \bigskip
  Another way of writing the ATT estimate that will prove useful is
  $$
    ATT \equiv \expec{Y_{i1}(1) - Y_{i1}(0)}{D_i = 1}
  $$

  \begin{itemize}
    \item The $\ |\ D_i = 1$ says we are averaging for the group whose treatment was $D_i = 1$
  \end{itemize}
\end{frame}

\begin{frame}{ATT Estimates}
  \vspace{-\bigskipamount}
  $$
    ATT \equiv \expec{Y_{i1}(1) - Y_{i1}(0)}{D_i = 1}
  $$

  \bigskip
  Why might the ATT be different than the ATE?

  \pause
  \bigskip
  The usual ``selection on $Y(1)$'' story
  \begin{itemize}
    \item Maybe units that benefit most from treatment select into it

    \pause
    \item But, not parallel trends rule out a lot of the $Y(0)$ stories
    \begin{itemize}
      \item E.g. Units can not select into treatment because of negative shocks
    \end{itemize}
  \end{itemize}
\end{frame}



\begin{frame}{``ATT'' in continuous DID}
  The concept of ATT vs. ATE is the same, but now there is no single `the treated'. Instead, there are two treated groups!

  There are now many different potential outcomes
  $$
    Y_{i1}(d),
  $$

  for $d = 0, 10, 20$
  \begin{itemize}
    \item What the outcome would be had the unit taken 0mg, 10mg, or 20mg

    \item We only observe on potential outcome $Y_{i1}(D_i)$
  \end{itemize}
\end{frame}

\begin{frame}{Treatment effects}
  We might care about the average effect of moving from 0 to some level $d > 0$:
  $$
    Y_{i1}(d) - Y_{i1}(0)
  $$

  \bigskip
  Or, we might want to know about the effect of increasing someone's dose from $d \to d'$
  $$
    Y_{i1}(d') - Y_{i1}(d)
  $$

  \begin{itemize}
    \item E.g. doctors wonder about increasing someone's dose to the next level
  \end{itemize}
\end{frame}

\begin{frame}{What did our two diff-in-diff estimates identify?}
  So what do we call the `ATT'? $\text{ATT}_{10}$ and $\text{ATT}_{20}$ perhaps?

  \bigskip\bigskip
  Callaway, Goodman-Bacon, and Sant'Anna call this:
  $$
    \text{ATT}(d \ \vert \ d) = \expec{Y_{i1}(d) - Y_{i1}(0)}{D_i = d}
  $$

  \begin{itemize}
    \item Our diff-in-diff's can only estimate the effect of a certain dose \emph{among the units that received that dose}.
  \end{itemize}
\end{frame}

\begin{frame}{$\text{ATT}(d \ \vert \ d)$}
  These dose-specific ATTs are identified from the same parallel trends assumption we are used to:
  \begin{itemize}
    \item Had the units receiving dose $D_i = d$ not received any dose $D_i = 0$, their trend in $Y$ would have evolved the same as the untreated units.
  \end{itemize}

  \bigskip
  The good news. We know how to think about the parallel trends assumption
  \begin{itemize}
    \item can potentially assess this assumption using pre-trends placebo checks
  \end{itemize}
\end{frame}

\begin{frame}{Comparing estimates}
  We observe a set of $\text{ATT}(d \ | \ d)$ for a set of doses (in our example $10$ and $20$)

  \bigskip
  The initial temptation is to compare these two estimates:
  \begin{itemize}
    \item E.g. finding the `correct' dose of a medicine by picking the highest $\text{ATT}$

    \item E.g. finding diminishing returns on an treatment investment

    \item E.g. what is the `per-unit' effect of the treatment?
  \end{itemize}
\end{frame}
%
% \begin{frame}{Thinking about counterfactual treatment effects}
%   We want to think about what would happen if one group of units received a \emph{different} dose?
%
%   \bigskip
%   $$
%     \text{ATT}(\tcbhighmath[colback = bgBlue]{d'} \ | \ \tcbhighmath[colback = bgPurple]{d}) =
%       \expec{Y_{i1}(\tcbhighmath[colback = bgBlue]{d'}) - Y_{i1}(0)}{D_i = \tcbhighmath[colback = bgPurple]{d}}
%   $$
%
%   % \devgrid
%   \begin{tikzpicture}[remember picture, overlay]
%     \node [anchor = south, text width = 0.4\textwidth] at (page cs:0.5,0.43) {
%       \begin{center}
%         \setstretch{0.8}
%         {\footnotesize\color{alice} Hypothetical dose}
%       \end{center}
%     };
%
%     \node [anchor = north, text width = 0.35\textwidth] at (page cs:0.75,0.48) {
%       \begin{center}
%         \setstretch{0.8}
%         {\footnotesize\color{purple} Dose they \emph{actually received}}
%       \end{center}
%     };
%   \end{tikzpicture}
% \end{frame}

\begin{frame}{Thinking about counterfactual treatment effects}
  We want to think about what would happen if one group of units received a \emph{different} dose?

  \bigskip
  One example is what Callaway, Goodman-Bacon, and Sant'Anna call the \alert{Average Causal Response}, $\text{ACRT}(d' \ | \ d)$:

  \bigskip\smallskip
  $$
    \text{ACRT}(\tcbhighmath[colback = bgBlue]{d'} \ | \ \tcbhighmath[colback = bgPurple]{d}) =
      \expec{\tcbhighmath[colback = bgBlue]{Y_{i1}(d' + \varepsilon) - Y_{i1}(d')}}{\tcbhighmath[colback = bgPurple]{D_i = d}}
  $$

  % \devgrid
  \begin{tikzpicture}[remember picture, overlay]
    \node [anchor = south, text width = 0.4\textwidth] at (page cs:0.56,0.52) {
      \begin{center}
        \setstretch{0.8}
        {\footnotesize\color{alice} Effect of increasing dose by a little}
      \end{center}
    };

    \node [anchor = north, text width = 0.35\textwidth] at (page cs:0.77,0.57) {
      \begin{center}
        \setstretch{0.8}
        {\footnotesize\color{purple} For the group that received $d$}
      \end{center}
    };
  \end{tikzpicture}
\end{frame}

\begin{frame}{Average Causal Response}
  \vspace*{-\bigskipamount}
  $$
    \text{ACRT}(\tcbhighmath[colback = bgBlue]{d'} \ | \ \tcbhighmath[colback = bgPurple]{d}) =
      \expec{\tcbhighmath[colback = bgBlue]{Y_{i1}(d' + 1) - Y_{i1}(d')}}{\tcbhighmath[colback = bgPurple]{D_i = d}}
  $$

  \begin{itemize}
    \item This measures the `per-unit' effect of the treatment (at some starting point $d'$)

    \item The effect is averaged over the group of units that received treatment $D_i = d$
  \end{itemize}

\end{frame}

\begin{frame}{Average Causal Response}
  \vspace*{-\bigskipamount}
  $$
    \text{ACRT}(\tcbhighmath[colback = bgBlue]{d'} \ | \ \tcbhighmath[colback = bgPurple]{d'}) =
      \expec{\tcbhighmath[colback = bgBlue]{Y_{i1}(d' + 1) - Y_{i1}(d')}}{\tcbhighmath[colback = bgPurple]{D_i = d'}}
  $$

  \bigskip
  Could imagine estimating this using our two $\text{ATT}(d \ | \ d)$:

  $$
    \text{ACRT}(d_1 \ | \ d_1) \approx \frac{\text{ATT}(d_2 \ | \ d_2) - \text{ATT}(d_1 \ | \ d_1)}{d_2 - d_1}
  $$
\end{frame}

\begin{frame}{Thinking about counterfactual treatment effects}
  Note that two things change whe we move from $d_1$ to $d_2$
  \begin{itemize}
    \item The dose amount they receive

    \item And the group of units we are estimating ATTs for
  \end{itemize}

  \bigskip
  The latter is what creates difficulty:
  \begin{itemize}
    \item If the group of units receiving treatment at $d_1$ is different than the group receiving treatment at $d_2$, then the ATTs are not directly comparable
  \end{itemize}

  \pause
  \bigskip
  Let's dig into this out with our example...
\end{frame}

\imageframe{lecture_includes/continuous_did/discrete_te_plot.pdf}

\begin{frame}{Homogeneous dose-ATTs}
  On the one extreme, we have homogeneity of treatment effects:
  $$
    \text{ATT}(d'  \ | \ d) = \text{ATT}(d')
  $$

  \begin{itemize}
    \item Treatment effects can depend on dose amount $d'$, but not on which treatment dose-group you are looking at
  \end{itemize}

  \pause
  \bigskip\bigskip
  This let's you directly compare $\text{ATT}(d'  \ | \ d')$ and $\text{ATT}(d \ | \ d)$ to estimate $\text{ACRT}(d' \ | \ d) = \text{ACRT}(d')$
\end{frame}

\imageframe{lecture_includes/continuous_did/discrete_te_linear_homogenous.pdf}
\imageframe{lecture_includes/continuous_did/discrete_te_linear_het.pdf}
\imageframe{lecture_includes/continuous_did/discrete_te_constant.pdf}
\imageframe{lecture_includes/continuous_did/discrete_te_maximizing.pdf}

% \begin{frame}{TWFE Regression}
%   What happens when you run the TWFE regression
%   $$
%     y_{it} = \tau D_{it} + \gamma_i + \delta_t + \epsilon_{it}
%   $$
%
%   \begin{itemize}
%     \item Regression compares people with larger and smaller doses of treatment
%
%     \item When making that comparison, you are not moving along any particular group's dose response curve
%   \end{itemize}
%
%   \bigskip
%   Assume this away (or that heterogeneity is `small')
% \end{frame}


\begin{frame}{Recap}
  So as a recap:

  \bigskip
  {\color{picton-blue} 1.} Dose-group specific DID estimates identify the $\text{ATT}(d \ | \ d)$ effects

  \bigskip
  {\color{picton-blue} 2.} Comparing across different doses does not inform us of the marginal effect of changing a dose, the $\text{ACR}(d \ | \ d)$
\end{frame}

\begin{frame}{``Strong Parallel Trends''}
  In Callaway, Goodman-Bacon, and Sant'Anna, they refer to an assumption called ``Strong Parallel Trends'' which combines two assumptions:
  \begin{enumerate}
    \item The parallel counterfactual trends assumption: each dose group has the same counterfactual trends as the untreated group
    \item Treatment effect homogeneity across groups: $\text{ATT}(d \ | \ d) = \text{ATT}(d)$
    \begin{itemize}
      \item $\implies \text{ACR}(d \ | \ d) = \text{ACR}(d)$
    \end{itemize}
  \end{enumerate}

  \pause
  \bigskip
  The latter is only needed to identify the $\text{ACR}$ curve using your $\text{ATT}(d \ | \ d)$ estimates
\end{frame}


\subsection{Continuous valued dose}

\begin{frame}{Doctor example extended}
  Now, let's say we have a continuous distribution of doses
  \begin{itemize}
    \item Some units receive 0mg
    \item Others receive some amount between 0 and 20mg
  \end{itemize}
\end{frame}

\imageframe{lecture_includes/continuous_did/continuous_raw_first_differences.pdf}

\begin{frame}{Dose-specific DID estimates}
  Our previous method of having dose-specific DID estimates will be quite noisy in this setting
  \begin{itemize}
    \item Very few individuals with the \emph{same} dose

    \item Throwing out information of individuals with very similar dose
  \end{itemize}
\end{frame}

\imageframe{lecture_includes/continuous_did/continuous_did_estimates.pdf}

\begin{frame}{Alternative strategies}
  Our previous strategy clearly will not suffice

  \bigskip
  An alternative strategy is to \emph{pool} observations with similar doses to estimate an average $\expec{\Delta Y_i}{D_i = d}$
  \begin{itemize}
    \item E.g. say you break doses into 2mg bins (0--2, 2--4, etc.) and take averages in those bins

    \item Or any other of your favorite non-parametric method for estimating
  \end{itemize}
\end{frame}

\begin{frame}{$\text{ATT}(d \ | \ d)$ estimators}
  In general, the $\text{ATT}(d \ | \ d)$ estimators will take the form of
  $$
    \expechat{\Delta Y_i}{D_i = d} - \expechat{\Delta Y_i}{D_i = 0}
  $$

  \begin{itemize}
    \item The second term is just the average change in $Y$ from the control group

    \item The first-term is where we have some options
  \end{itemize}
\end{frame}

\begin{frame}{Option 1: Bins}
  The simplest procedure is to `discretize' doses into a set of bins

  \bigskip
  Then, form a DID estimate for each bin
\end{frame}

\imageframe{lecture_includes/continuous_did/continuous_binned_avgs.pdf}

\begin{frame}{Pros and Cons of Bins}
  Advantges:
  \begin{itemize}
    \item Creates a discrete set of groups that are easy to discuss in your paper

    \item Event-studies are easy to estimate (one for each bin)
  \end{itemize}

  \bigskip
  Disadvantages:
  \begin{itemize}
    \item Bins create a flat surface that is likely not perfectly specified (but gets close as the number of bins grow)
  \end{itemize}
\end{frame}

\begin{frame}{Option 2: Non-parametric smoothing}
  Alternatively, $\expechat{\Delta Y_i}{D_i = d}$ can be estimated with non-parametric methods

  \bigskip
  The simplest way is to use non-parametric method on $\Delta Y_i - \expechat{\Delta Y_i}{D_i = 0}$ with only the $D_i > 0$ observations.
  \begin{itemize}
    \item The result of this will be the DID estimates

    \item Standard errors produced should be correct (can always bootstrap to be safe)
  \end{itemize}
\end{frame}

\imageframe{lecture_includes/continuous_did/continuous_smoothed_avgs.pdf}

\begin{frame}{Pros and Cons of Bins}
  Advantges:
  \begin{itemize}
    \item Better estimates the smooth evolution of $\expechat{\Delta Y_i}{D_i = d}$
  \end{itemize}

  \bigskip
  Disadvantages:
  \begin{itemize}
    \item Event-studies are hard to display (one plot for each event-time?)
  \end{itemize}
\end{frame}


\section{Empirical Example}

\begin{frame}{Lu and Yu, 2015}
  This example will use data from
  ``Trade Liberalization and Markup Dispersion: Evidence from China's WTO Accession'' (Lu and Yu, American Economic Journal: Applied Economics 2015)
\end{frame}

\begin{frame}{China's WTO Ascension}
  The paper uses the entrance of China into the World Trade Organization (WTO) in 2002

  \bigskip
  When entering the WTO, tariffs at the industry level were expected to drop to at most 10\%
\end{frame}

\begin{frame}{Defining continuous treatment variable}
  The research strategy is to compare industries (at the 3-digit SIC level) that had higher or lower pre-existing tariff rates

  \begin{itemize}
    \item After the entrance in 2002, industries with higher tariff had \emph{larger} decreases in tariffs $\implies$ a larger shock in import competition
  \end{itemize}

  \bigskip
  Can define treatment as
  $D_i = \max (0, \text{Tariff}_{2001} - 0.10)$
  \begin{itemize}
    \item How much tariffs are expected to drop
  \end{itemize}
\end{frame}

\begin{frame}{Outcome variable: Markup Dispersion}
  The dataset is an industry-by-year panel dataset
  \begin{itemize}
    \item Each row is an industry in China in a given year
  \end{itemize}

  \bigskip
  The outcome variable is the industry-level \emph{Theil Markup Index}, which measures how much spread in markups at the firm level
  \begin{itemize}
    \item Higher markup index $\implies$ less competitive market
  \end{itemize}

  \pause
  \bigskip
  \textbf{Hypothesis:} Larger decrease in tariffs $\implies$ more competition $\implies$ decrease in markup dispersion
\end{frame}

\begin{frame}{$2 \times 2$ setup}
  We will use 2000 ($t = 0$) and 2004 ($t = 1$) as our initial years

  \bigskip
  Let's look at the raw data
  \begin{itemize}
    \item $x$-axis is our treatment vairable $D_i = \max (0, \text{Tariff}_{2001} - 0.10)$

    \item $y$-axis is the change in markup dispersion $\Delta \ln \text{Theil}_i = \ln \text{Theil}_{i, 2004} - \ln \text{Theil}_{i, 2000}$
  \end{itemize}
\end{frame}

\imageframe{lecture_includes/continuous_did/china_wto_raw.pdf}

\begin{frame}{Parallel Trends assumption}
  Our identification assumption is that if China did not enter the WTO, counterfactual changes in markup dispersion is the same for the unaffected group and the affected doses

  \begin{itemize}
    \item We can assess the plausability using pre-periods (later)
  \end{itemize}

  \pause
  \bigskip\bigskip
  To estimate counterfactual trend, we will take average $\Delta \ln \text{Theil}_i$ for the $D_i = 0$ group
\end{frame}

\imageframe{lecture_includes/continuous_did/china_wto_raw.pdf}

\imageframe{lecture_includes/continuous_did/china_wto_ests.pdf}

\begin{frame}{Pre-trends `test'}
  Redo the previous estimates using
  $$
    \Delta \ln \text{Theil}_i = \ln \text{Theil}_{i, 1998} - \ln \text{Theil}_{i, 2000}
  $$

  \bigskip
  We are hoping that the pre-trend change in markup dispersion is uncorrelated with the dose received, $D_i$
\end{frame}

\imageframe{lecture_includes/continuous_did/china_wto_pretrends_ests.pdf}

\end{document}
